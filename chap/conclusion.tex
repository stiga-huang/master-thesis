% Copyright (c) 2014,2016 Casper Ti. Vector
% Public domain.

\specialchap{总结与展望} \label{chap:conclusion}

在大数据背景下,随着分布式图处理系统在金融、电信、刑侦等行业的应用,图数据库系统迎来了新的挑战。在这些应用场景中,图模型里的边表示的是两个实体之间事件类型的关系,如通话关系、交易关系等,这使得两点之间会连接着大量的重边。传统的图数据库在处理大量重边时性能会急剧下降,具体表现在邻域点集边集相关的查询上。对于依赖邻域点集查询实现的高级查询,其查询耗时更会呈指数级增长。

Titan是目前应用广泛的开源分布式图数据库,其基于HBase、 Cassandra 或 BerkerlyDB 实现存储。图数据库Titan将图数据以邻接表的方式存储在BigTable模型中,在处理大量重边时也有性能问题,主要有两方面的原因:
\begin{enumerate}
    \item 不管基于BigTable实现中的列修饰符如何设计,总可以找到限定条件不包含列修饰符首位的查询,使得查询需要遍历整行邻接表
    \item 缓存中存放了大量与查询结果无关的重边数据,增加了缓存的失效率
\end{enumerate}
其本质原因是所有重边数据都存储在邻接表中,使得处理大量重边时,邻接表规模急剧膨胀。

本文根据以上分析,提出了一种复合存储架构——HyBriG,基于Titan和HBase建立存储层,用Titan来存储图的结构信息和点集的属性信息,HBase存储边集的所有属性信息。在HybriG中邻接表保持了项数和数据量上的精简,从而能克服上述图数据库的缺点。相比于传统图数据库Titan的实现,HyBriG在处理大量重边时拥有更优异的查询性能,在数据导入方面也有不错的表现。

在获得高性能的同时,所带来的牺牲就是简化了对事务性的支持,因为将图数据分开存储在Titan和HBase中带来了两个系统间数据一致性的问题。HybriG在对事务的支持中进行了折衷,没有实现严格的事务性支持。HybriG通过合理规划读写来避免冲突,并保证数据在批量导入时两系统间的最终一致性。事实上,在许多含有大量重边的应用场景中,数据都是收集得来的事实数据(事件数据),因此数据都是定期批量导入的,除此之外都是只读操作,导入的数据不需要修改,因此并不需要很强的事务性支持。HybriG架构提供了最终一致性,足以满足这些场景的应用需求。

在实践中,明略数据的关联数据分析平台——SCOPA基于HybriG架构开发了核心存储部件。证明了HybriG架构在处理大量重边时的优异性能。

\vspace{3mm}

本文的主要贡献在于:
\begin{itemize}
  \item 研究并分析了图数据库Titan在处理大量重边时查询性能下降的原因
  \item 基于上述分析提出了一种能高效处理大量重边的属性图存储架构HyBriG
  \item 通过实验验证了HyBriG架构性能的优越性
\end{itemize}

关于进一步的研究方向,在HybriG之上还可以添加分布式的全图数据统计等OLAP的支持,也可以添加图计算的支持。因为HybriG把所有重边数据都存储在HBase中,而且这些数据在导入之后就是只读的,因此可以利用HBase的Snapshot技术对数据定期建立快照,此后在HBase Snapshot上利用MapReduce、Spark等计算框架即可实现全图数据的批量计算。由于HBase Snapshot直接是HDFS上的文件,因此对它们进行分析的性能会远优于通过HBase RegionServer读取数据再进行分析。在这方面HybriG相比传统图数据库Titan具有先天优势。

在HybriG之上也可以实现简化的事务性支持,即对于图中的点的操作提供事务支持。因为在HybriG中,点的所有数据都是存储在Titan中的,而Titan提供了事务性支持,因此我们可以基于此实现点集数据操作的事务性支持。另外我们也可以尝试在Titan和HBase上实现分布式事务来提供完整的事务性支持,以支持一些场景下需要修改数据的需求。这些都是可以扩展的研究方向。

% vim:ts=4:sw=4
