% Copyright (c) 2014,2016 Casper Ti. Vector
% Public domain.

\begin{cabstract}
	在图中,起点和终点都相同的两条边称为重边。属性图是一种带标志和重边的有向图,图中的点和边可以拥有任意数目的属性值。属性图由于其丰富的表达能力而广泛应用于实际建模中。实际应用中一般用图数据库解决属性图的存储需求。相比于传统的关系型数据库,图数据库在做多跳邻域查询、路径查询等与图结构相关的查询时,具有更优异的性能。Titan是产业界日渐关注的一个开源的分布式图数据库,Titan的数据以邻接表的方式组织,每个点的邻接表存储了相邻的所有边,这使得与邻接点集相关的查询都需要遍历整个邻接表。当图中含有大量重边时,邻接表规模巨大,这种数据组织方式导致邻域查询性能严重受损。而邻域查询是大部分图查询的基础,如多跳邻域查询、路径查询、局部聚集系数查询(计算)等,这些查询往往由嵌套的邻域查询实现,随着邻域深度的增加,这种性能受损将被急剧放大。本文提出了一种基于Titan和列式存储数据库HBase的复合架构设计——HybriG,基于Titan和HBase建立存储层,用Titan来存储图的结构信息和点集的属性信息,HBase存储边集的所有属性信息。在HybriG中邻接表保持了项数和数据量上的精简,从而能克服上述图数据库的缺点。相比于传统图数据库Titan,HybriG在邻域点集相关查询以及边集数据批量导入上的性能提升一个量级以上。本文介绍了HybriG基于Titan和HBase的存储设计,并描述了在此存储设计基础上,如何高效地实现图查询以及图数据的插入操作。此外,本文还提出了图数据的高效导入方案,并保证导入过程中Titan与HBase存储数据的一致性。最后通过实验验证了HybriG在处理大量重边时的优异性能。
\end{cabstract}

\begin{eabstract}
	The past decades have witnessed the massive growth of Internet. Vast amount of graph data was produced by the boom of social network, e-commerce and online education. To analyze and manage graph data, there are two branches of development. One focus on distributed graph computing. Solving problems like PageRank or other algorithms that fit into the Bulk Synchronous Parallel (BSP) model. Systems like Pregel and GraphLab are proposed for this branch. The other branch focus on management of graph data, providing support like OLTP and graph queries. In this branch, graph databases like Neo4j and Titan are developed for management of property graph.

	The property graph is a directed, labeled graph with multi-edges, i.e., edges with the same source vertex and destination vertex. Vertices and edges can be associated with any number of properties. Since it can represent graph data in most scenarios, the property graph model has numerous applications in industry. 

	However, traditional graph databases encounter significant performance degradation when the graph contains large amount of multi-edges. 
	We reveal the cause of this in Titan. It’s an open source distributed graph database, which has attracted wide attention in industry. Furthermore, we propose HybriG, a better distributed architecture based on Titan and HBase for this scenario. 

	Titan stores graphs in adjacency list format, where a graph is stored as a collection of vertices with their adjacency lists. Each entry of the adjacency list stores an edge or a vertex property. 
	When querying about adjacent vertices, Titan has to look through the entire adjacency list of the source vertex, which is a waste since we just need the connected vertices but not the multi-edges. This cost hurts the performance when the multi-edge set grows explosively. For high level queries who bases on adjacent vertices query, e.g. path queries, local cluster coefficient queries etc., the situation will be worse. 
	HybriG implements property graph model as well. It stores the vertex data and graph structure in Titan, the edge data in HBase, respectively. We chose HBase since it’s one of the storage engines of Titan and is widely used in industry. Storing part of the graph data into HBase won’t bring too much cost because it’s also the data store of Titan.
	This separation helps HybriG to keep a concise adjacency list about the graph structure, which helps to gain an order of magnitude improvement in execution of adjacent vertices based queries and batch loading of edge set. The difficulty of this separation solution is that we should guarantee the consistency of edge data between Titan and HBase. 
	In most of the scenarios with large multi-edges set, multi-edges are representing event-like data, e.g. phone call records between two people, which won’t be modified after insertion. Thus we can relax the consistency constrain of edge data to just guarantee consistency for insertion. We leverage the transaction in Titan to achieve consistency with HBase in edge insertion, especially in batch loading of edge data. Finally, extensive experiments have been conducted to show the outstanding performance of HybriG.
\end{eabstract}

% vim:ts=4:sw=4
