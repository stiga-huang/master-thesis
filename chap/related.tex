% Copyright (c) 2014,2016 Casper Ti. Vector
% Public domain.

\chapter{相关工作}
图数据分析与处理是大数据背景下的一大应用分支\supercite{big_data},大数据背景下的图处理系统可以分为图计算框架和图数据库两大类。
图计算框架旨在高效地进行全图量级的并行计算,如计算PageRank\supercite{pagerank}、社区发现\supercite{community_detection}(Community Detection)、子图匹配\supercite{subgraph_listing}等,提供的是对OLAP(Online Analytical Processing)的支持。这类系统中,Pregel\supercite{pregel}、GraphLab\supercite{graphlab}、PowerGraph\supercite{powergraph}支持BSP\supercite{BSP}模型的计算,对图数据进行特定的分割,使得集群中的机器可以在内存中计算分区的数据,再通过交互完成全量数据的计算。上述系统中,Pregel没有开源,GraphLab和PowerGraph都需要单独部署,无法有机地融入已有的大数据生态系统中,需要导出数据以供计算。GraphX\supercite{graphx}解决了这个问题,其是基于分布式内存计算框架Spark\supercite{spark}实现的图计算引擎,使得图计算可以融入整个数据流处理的框架。相对于分布式计算框架,GraphChi\supercite{graphchi}则追求在单机完成大规模图数据的计算问题。在图计算领域还有许多研究旨在提高图计算框架在特定问题的处理性能,如\supercites{xuning_LogGP,xuning_2,shaoxia_1,shaoxia_2,shaoxia_3}等。学术界对图计算的研究热情普遍高涨。
图数据库专注于图数据的管理,提供高效的图遍历查询(graph traversal)。图数据模型则是指图数据库如何以图的方式对数据进行抽象 \supercite{graph_models_survey},应用较广泛的图数据模型有RDF 模型和属性图模型。RDF全称为Resource Description Framework,是由W3C制定的知识描述标准,使得按RDF表示的不同数据源可以进行数据交换或合并。RDF中的数据单元是由主语、谓语、宾语组成的三元组,可以直接对应为图上的一条边,因此将RDF模型归类为图数据模型。常见的RDF存储有Jena 、AllegroGraph 等。当今的图数据库大都采用属性图模型设计\supercite{graph_database_models},如DEX\supercite{DEX}、GraphChi-DB\supercite{graphchi-db}、Neo4j、Titan等,其中DEX和GraphChi-DB都只是单机数据库,Neo4j的开源版本也是单机的,只适合处理中等规模的图。Titan的底层实现了可插拔的存储接口,可部署在HBase、Cassandra或BerkeleyDB之上,因此可以很好地与Hadoop集群结合,组成统一的数据处理框架。然而,现有的图数据库都没有考虑含有大量重边的属性图应用场景。HybriG架构填补了这个空白,为含有大量重边的属性图应用场景提供了一种可行的解决方案。


% vim:ts=4:sw=4
