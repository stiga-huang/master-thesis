% Copyright (c) 2014,2016 Casper Ti. Vector
% Public domain.

\chapter{致谢}

匆匆七年,就要挥手告别大学时光了。看着理一楼下的银杏又长满了绿叶,下一个赏叶的金秋自己就不在校园里了。突然有点后悔,没有过好恣意赏叶的时光。北大太美,有太多美景没来得及细品,也有太多事情没来得及完成。然而,岁月的齿轮不会回转,我们能做的终究是珍惜当下,走好未来的路。

感谢燕园赠与了我七年的美好时光。这里是一片奋斗的故土,永远忘不了通宵后松林六点钟的早餐,还有刷夜时凌晨三点的手抓饼摊。这里也是成长的地方,图书馆的走廊,邱德拔的球场,计算机系的机房,还有太多的地点,带给了我无数快乐的时光。每每走过,还能想起当初的那个自己,有过幼稚,有过自大,也有过彻悟。七情六欲,好像都能在校园里找到回忆。

最想感谢的人,是我的导师崔斌教授。或许由于缘份,大一时我的指导老师就是崔老师。崔老师治学严谨,在学术上有很强的洞察力,总能捕捉到学术前沿的发展方向,能在崔老师的指导下做研究是我们的荣幸。感谢老师的悉心指导,这些年我松懈过也努力过,感谢老师的宽容和期待。回想本科的四年,我只是在计算机领域里打了点基础;研究生的三年,才让我真正找到了专长,也愿意在这个方向继续努力下去,这些都离不开老师的指引。崔老师在学术上的孜孜不倦,将一直是我们学习的榜样。
% 还记得老师提到的全年360*10小时工作。我时常想到,追求事业就该如此。

感谢实验室的兄弟姐妹们。感谢大牛师兄姚俊杰,在我初入实验室时的悉心指导。感谢大师姐徐赢,还有土豪师姐谢怡然,小师姐黄艳香,是你们让实验室无比温馨。感谢优秀网管陈学轩师兄,以及为人随和谦逊的陈琛师兄,在读研之初,我就希望自己做人做事,能和你们相当。感谢在学术上孜孜不倦,辛苦耕耘的志师兄、邵侠、施老板和徐宁师兄,以及将要毕业的乐乐、佳伟,一直敬佩你们博士生的学术追求,从你们身上我学到了很多。感谢乐观又积极开朗的智鹏、沉着又不乏幽默的羚宇,你们为实验室带来了许多欢声笑语。还有一同度过01时光的琪姐、dk、光哥,以及本科时1631的胡志挺、陈宇望、杜焱,我会永远铭记那些美好的时光。感谢实验室的闫学灿、李旭鹏、陈一茹、苑斌、谢旭、符芳诚、李恬等各位师弟师妹,还有好多实验室的同学,不及一一细说。

感谢实习中遇到的各位同事。感谢腾讯的导师曹坤,带我养成了许多好习惯,也带领我见识到了工业界的优秀项目。感谢大师konton、Java大神大桂哥、炫酷风趣的Justy、极富学术追求的徐钊,还有精准推荐组的各位同事,暑期实习的两个月让我收获良多。感谢明略数据的冯博冯是陪、董哥董斌一直以来的看重。感谢史上最nice的leader孟嘉师兄、相声口才的任鑫琦师兄,感谢你们的信任,让我在项目中得到了真正的锻炼。感谢一起封闭开发过的同事,自带东北幽默的石海洋、爽快不羁的大哥周扬、文艺青年付骁弈、半夜打游戏但开发效率超神的李博龙,以及靠谱运维一把手闫强,我们一同经历了封闭开发,和你们合作让我收获了很多。感谢后来一起合作过的队长王啸风、为人超nice的刘泉海、老司机杨洋、有为小青年朱亚超。


最后还要感谢我的家人,感谢你们在我求学生涯里一如既往的关怀和支持。感谢我的女朋友许艺萱,我们一起成长,一起追求美好的未来。

每一段时光终会划上一个句号,感谢所有在我世界中出现过的人们,我们的人生曾有意无意地互相影响过,对此我永远心怀感恩。

% vim:ts=4:sw=4
